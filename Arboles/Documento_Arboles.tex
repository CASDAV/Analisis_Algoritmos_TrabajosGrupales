\documentclass[10pt,letterpaper]{article}
\usepackage[utf8]{inputenc}
\usepackage[spanish]{babel}
\title{Arboles Binarios}
\author{David Gutierrez Alarcon,\\ Julian Andres Carrillo Chiquisa,\\ David Alejandro Castillo Chíquiza}
\begin{document}
\maketitle

\part*{Complejidad computacional}
	
	Al ser arboles binarios la complejidad para llegar a un nodo en el peor de los casos es de $O(n)$, para el árbol generado por Huffman el peor de los casos siempre sera la regla para encontrar el valor buscado, sin embargo para el generado por el optimo eso puede variar, ya que sus nodos intermedios también valores que podrían llegar a ser buscados, ya que los valores mas frecuentes se encuentran mas cerca de la raíz en la mayoría de los casos no es necesario que baje hasta encontrar un nodo hoja. 

\part*{Longitud del mensaje}
Para realizar la comparación hicimos uso de un fragmento del quijote 2, este estaba compuesto por aproximadamente por 3300 palabras.

	\subsection*{Huffman}

	El archivo generado por medio de este algoritmo pesa 14Kb y tiene una longitud 13619 caracteres.

	\subsection*{Optimo}
	
	El archivo generado por medio de este algoritmo pesa 9Kb y tiene una longitud 9173 caracteres.
	
	
\end{document}