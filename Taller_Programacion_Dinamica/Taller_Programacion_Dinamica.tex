\documentclass[]{article}
\usepackage[utf8]{inputenc}
\usepackage[spanish]{babel}
\usepackage{float}
\usepackage{amsmath}
\usepackage{amsthm}
\usepackage{amssymb}
\PassOptionsToPackage{normalem}{ulem}
\usepackage{ulem}

\newcommand{\noun}[1]{\textsc{#1}}
\floatstyle{ruled}
\newfloat{algorithm}{tbp}{loa}
\providecommand{\algorithmname}{Algorithm}
\floatname{algorithm}{\protect\algorithmname}

\numberwithin{equation}{section}
\numberwithin{figure}{section}
\theoremstyle{definition}
\newtheorem*{defn*}{\protect\definitionname}

\usepackage{xmpmulti}
\usepackage{algorithm,algpseudocode}

\makeatother

\addto\shorthandsspanish{\spanishdeactivate{~<>}}

\addto\captionsenglish{\renewcommand{\definitionname}{Definition}}
\addto\captionsenglish{\renewcommand{\algorithmname}{Algoritmo}}
\addto\captionsspanish{\renewcommand{\definitionname}{Definición}}
\providecommand{\definitionname}{Definition}

\begin{document}
\title{Taller 2 - Programación Dinámica}
\author{\selectlanguage{spanish}%
David Gutierrez Alarcon,\\ Julian Andres Carrillo Chiquisa,\\ David Alejandro Castillo Chíquiza}
\date{\selectlanguage{spanish}%
\today}
\maketitle

\begin{abstract}
En este documento se presenta el análisis de los dos problemas planteados y sus soluciones mediante el uso de el método de Dividir y Vencer
\end{abstract}

\part*{Análisis y Diseño del Problema}

\section*{Análisis}


\section*{Diseño}


\part*{Algoritmos}


\section*{Evidente recursivo}
	
	\subsection*{Pseudocódigo}
	
	\begin{algorithm}[H]
	\begin{algorithmic}[1]

	\Procedure{EvidenteRecursivo}{$X,Y,Z,A,Q$}

  		\If{$A == Q$}
  			\State $s \leftarrow True$
  			\State\Return $A$
  		\EndIf
  		
  		\If {$s == False$}
  		
  			\If{$(longitud(X)==0) \wedge (longitud(Y) == 0)$}
  				\State\Return $A$
  			\EndIf
  			
  			\If {$(logitud(X) > 0) \vee (longitud(Y) > 0)$}
  				\If{$longitud(X) > 0$}
  					\If{$X[0] == Z[0]$}
  						\State $EvidenteRecursivo(X[1:],Y,Z[1:],A+X[0],Q) $
  					\EndIf
  					
  				\EndIf
  				\If{$longitud(Y) > 0$}
  					\If{$Y[0] == Z[0]$}
  						\State $EvidenteRecursivo(X,Y[1:],Z[1:],Z+Y[0],Q)$
  					\EndIf
  				\EndIf
  			\EndIf
  		\EndIf
  		\State\Return $A$

	\EndProcedure

	\end{algorithmic}
	\caption{\foreignlanguage{english}{EvidenteRecursivo}}
	\end{algorithm}
	
	
	\subsection*{Complejidad}
	
	\subsection*{Invariante}
	
	\subsection*{Notas de Implementación}

\section*{Memorizado}

	\subsection*{Pseudocódigo}
	
	\subsection*{Complejidad}
	
	\subsection*{Invariante}
	
	\subsection*{Notas de Implementación}

\section*{Bottom-up}

	\subsection*{Pseudocódigo}
	
	\subsection*{Complejidad}
	
	\subsection*{Invariante}
	
	\subsection*{Notas de Implementación}

\part*{Comparación}


\end{document}