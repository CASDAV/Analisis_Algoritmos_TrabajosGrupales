\documentclass[]{article}
\usepackage[utf8]{inputenc}
\usepackage[spanish]{babel}
\usepackage{float}
\usepackage{amsmath}
\usepackage{amsthm}
\usepackage{amssymb}
\PassOptionsToPackage{normalem}{ulem}
\usepackage{ulem}

\newcommand{\noun}[1]{\textsc{#1}}
\floatstyle{ruled}
\newfloat{algorithm}{tbp}{loa}
\providecommand{\algorithmname}{Algorithm}
\floatname{algorithm}{\protect\algorithmname}

\numberwithin{equation}{section}
\numberwithin{figure}{section}
\theoremstyle{definition}
\newtheorem*{defn*}{\protect\definitionname}

\usepackage{xmpmulti}
\usepackage{algorithm,algpseudocode}

\makeatother

\addto\shorthandsspanish{\spanishdeactivate{~<>}}

\addto\captionsenglish{\renewcommand{\definitionname}{Definition}}
\addto\captionsenglish{\renewcommand{\algorithmname}{Algoritmo}}
\addto\captionsspanish{\renewcommand{\definitionname}{Definición}}
\providecommand{\definitionname}{Definition}

\begin{document}
\title{Taller 2 - Programación Dinámica}
\author{\selectlanguage{spanish}%
David Gutierrez Alarcon,\\ Julian Andres Carrillo Chiquisa,\\ David Alejandro Castillo Chíquiza}
\date{\selectlanguage{spanish}%
\today}
\maketitle

\begin{abstract}
En este documento se presenta el análisis de los dos problemas planteados y sus soluciones mediante el uso de el método de Programación Dinamica
\end{abstract}

\part*{Análisis y Diseño del Problema}

\section*{Análisis}

\text Para el desarrollo del ejercicio, es necesario que se dispongan de dos cadenas $X$ y $Y$ de m y n caracteres respectivamente, con el fin de determinar si el algoritmo es capaz de barajar las dos  secuencias de elementos anteriores, de una forma determinada por una cadena $Z$ dada previamente,esto implica que la cadena barajada debe estar conformada tomando todos los elementos de $X$ y $Y$, con orden de $Z$, si embargo, no necesariamente deben ser contiguos. El problema se puede modelar de la siguente manera: 

$$X=\langle x_1,x_2,...,x_m \rangle$$
$$Y=\langle y_1,y_2,...,y_n \rangle$$
$$Z=\langle z_1,z_2,...,z_{m+n} \rangle = \langle z_i \in T \hspace{1mm} 1<i \leq n+m \rangle$$

\text Donde n y m son la cantidad respectiva de los conjuntos $X$ y $Y$, dando a entender que $n+m$ son todos los elementos de $Z$ con $z_i$ como los elementos pertenecientes al conjunto $T$.


\section*{Diseño}

\text El Algoritmo debe validar si es posible generar una secuencia con los elmentos de $X$ y $Y$ que correspondan con los elementos de $Z$.

\subsection*{Redursivo Evidente}

\subsubsection*{Entradas}

\begin{itemize}

\item Una Secuencia $X=\langle x_1,x_2,...,x_m \rangle$
\item una secuencia $Y=\langle y_1,y_2,...,y_n \rangle$  
\item Una secuencia $Z=\langle z_1,z_2,...,z_{m+n} \rangle = \langle z_i \in T \hspace{1mm} 1<i \leq n+m \rangle$
\item Una secuencia $A$ vacia
\item Una secuencia copia de $Z$ de nombre $Q$

\end{itemize}

\subsubsection*{Salidas}

\begin{itemize}

\item Un dato Booleano que determina si es posible la generación de la secuencia $Z$

\end{itemize}

\subsection*{Memoizado}

\subsubsection*{Entradas}

\begin{itemize}

\item Una Secuencia $X=\langle x_1,x_2,...,x_m \rangle$
\item una secuencia $Y=\langle y_1,y_2,...,y_n \rangle$  
\item Una secuencia $Z=\langle z_1,z_2,...,z_{m+n} \rangle = \langle z_i \in T \hspace{1mm} 1<i \leq n+m \rangle$
\item Valor $m$ que hace referencia al tamaño de la longitud de $X$
\item Valor $n$ que hace referencia al tamaño de la longitud de $Y$
\item Matriz $C$ vacia de tamaño $m,n$

\end{itemize}

\subsubsection*{Salidas}

\begin{itemize}

\item Secuencia de Carcacteres correspondientes a $Z$ en el caso dado donde si sea posible generar la secuencia de caracteres, en caso contrario, se presenta una secuencia con la mayor cantidad de caracteres posibles similares a $Z$

\end{itemize}

\subsection*{Bottom-up}

\subsubsection*{Entradas}

\subsubsection*{Salidas}

\part*{Algoritmos}

\section*{Evidente recursivo}
	
	\subsection*{Pseudocódigo}
	
	\begin{algorithm}[H]
	\begin{algorithmic}[1]

	\Procedure{EvidenteRecursivo}{$X,Y,Z,A,Q$}

  		\If{$A == Q$}
  			\State $s \leftarrow True$
  			\State\Return $A$
  		\EndIf
  		
  		\If {$s == False$}
  		
  			\If{$(longitud(X)==0) \wedge (longitud(Y) == 0)$}
  				\State\Return $A$
  			\EndIf
  			
  			\If {$(logitud(X) > 0) \vee (longitud(Y) > 0)$}
  				\If{$longitud(X) > 0$}
  					\If{$X[0] == Z[0]$}
  						\State $EvidenteRecursivo(X[1:],Y,Z[1:],A+X[0],Q) $
  					\EndIf
  					
  				\EndIf
  				\If{$longitud(Y) > 0$}
  					\If{$Y[0] == Z[0]$}
  						\State $EvidenteRecursivo(X,Y[1:],Z[1:],Z+Y[0],Q)$
  					\EndIf
  				\EndIf
  			\EndIf
  		\EndIf
  		\State\Return $A$

	\EndProcedure

	\end{algorithmic}
	\caption{\foreignlanguage{english}{EvidenteRecursivo}}
	\end{algorithm}
	
	
	\subsection*{Complejidad}
	
	\text Para este algoritmo, se tiene que revisar todas las combinaciones de las secuencias $X$ y $Y$, por lo tanto la complejidad es $O(2^{mn})$
	
	\subsection*{Invariante}
	
	\text Todos los elementos de $X$ y $Y$ se agregan a la lista $A$, cumpliendo con la propiedad de orden y pueden o no estar en un orden consecutivo
	
	\subsubsection*{inicio}
	\begin{itemize}
	\item Lista $A$ vacia
	\end{itemize}
	\subsubsection*{Avance}
	\begin{itemize}
	\item Si el primer item del arreglo $X$ ó $Y$ coincide con el primer elemento de $Z$, dicho elemnto se agrega a la lista de $A$
	\end{itemize}
	\subsubsection*{Terminación}
	\begin{itemize}
	\item Termina cuando la lista de $A$ es igual a la lista de $Q$ y retorna verdadero
	\item Termina cuando no se encuentren caracteres en $X$ y $Y$ que coincidan con los carcateres de la lista $Z$ y retorna Falso
	\end{itemize}
	
	\subsection*{Notas de Implementación}
	
	\text La implementación del pseudocodigo fue realizada en el lenguaje Python y se puede encontrar en el archivo adjunto a este documento, El nombre del archivo es: recursivoInocente.ipynb.

\section*{Memoizado}

	\subsection*{Pseudocódigo}
	
	\begin{algorithm}[H]
	\begin{algorithmic}[1]
	
	\Procedure{Memoizado}{$X,Y,Z,M,N,C$}
	
	\If{$C[M][N] != ""$}
		\State\Return $C[M][N]$
	\EndIf
	
	\If{$longitud(z) == 0$}
		\State $sePudo = True$
		\State\Return $C[M][N]$
	\EndIf
	
	\If{$longitud(X) > 0 \vee longitud(Y) > 0$}
		\If{$longitud(X) > 0$}
			\If{$X[0] == Z[0] \wedge sePudo == False$}
				\State $C[M][N] = str(Memoizado(X[1:],Y,Z[1:],M-1,N,C))+X[0]$
			\EndIf
		\EndIf
		
		\If{$longitud(Y) > 0 $}
			\If{$Y[0] == Z[0] \vee sePudo == False$}
				\State $C[M][N] = str(Memoizado(X,Y[1:],Z[1:],M,N-1,C))+Y[0]$
			\EndIf
		\EndIf
	\EndIf
	
	\State\Return $C[M][N]$	
	
	\EndProcedure
	
	\end{algorithmic}
	\caption{\foreignlanguage{english}{Memorizado}}
	\end{algorithm}
	
	\subsection*{Complejidad}
	
	 \text La complejidad se determina teniendo en cuenta que se genera una matriz de tamaño $m,n$ y haciendo uso de la Memiozación podemos decir que la complejidad del algoritmo en $O(n^2)$
	
	\subsection*{Invariante}
	
	\subsubsection*{inicio}
	\begin{itemize}
	\item Matriz $C$ vacia
	\end{itemize}
	\subsubsection*{Avance}
	\begin{itemize}
	\item Si la matriz en la posicion $m,n$ no está vacia, retorna el contenido en dicha posición. 
	\item Si el primer item del arreglo $X$ ó $Y$ coincide con el primer elemento de $Z$, dicho elemnto se agrega a la matriz $C$ en la posición $m,n$.
	\end{itemize}
	\subsubsection*{Terminación}
	\begin{itemize}
	\item Termina cuando el arreglo de $Z$ está vacio y retorna la matriz $C$ en la posición $m,n$.
	\item Termina cuando no se encuentren caracteres en $X$ y $Y$ que coincidan con los carcateres de la lista $Z$ y retorna la matriz $C$ en la posición $m,n$.
	\end{itemize}
	
	\subsection*{Notas de Implementación}
	
	\text La implementación del pseudocodigo fue realizada en el lenguaje Python y se puede encontrar en el archivo adjunto a este documento, El nombre del archivo es: recursivoMemoizado.py

\section*{Bottom-up}

	\subsection*{Pseudocódigo}
	
	\subsection*{Complejidad}
	
	\subsection*{Invariante}
	
	\subsubsection*{inicio}
	\subsubsection*{Avance}
	\subsubsection*{Terminación}
	
	\subsection*{Notas de Implementación}
	
	

\part*{Comparación}


\end{document}