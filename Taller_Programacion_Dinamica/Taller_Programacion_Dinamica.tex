%% LyX 2.3.6.1 created this file.  For more info, see http://www.lyx.org/.
%% Do not edit unless you really know what you are doing.
\documentclass[spanish,english]{article}
\usepackage[T1]{fontenc}
\usepackage[utf8]{inputenc}
\usepackage{float}
\usepackage{amsmath}
\usepackage{amsthm}
\PassOptionsToPackage{normalem}{ulem}
\usepackage{ulem}

\makeatletter

%%%%%%%%%%%%%%%%%%%%%%%%%%%%%% LyX specific LaTeX commands.
\newcommand{\noun}[1]{\textsc{#1}}
\floatstyle{ruled}
\newfloat{algorithm}{tbp}{loa}
\providecommand{\algorithmname}{Algorithm}
\floatname{algorithm}{\protect\algorithmname}

%%%%%%%%%%%%%%%%%%%%%%%%%%%%%% Textclass specific LaTeX commands.
\theoremstyle{definition}
\newtheorem*{defn*}{\protect\definitionname}

%%%%%%%%%%%%%%%%%%%%%%%%%%%%%% User specified LaTeX commands.
\usepackage{xmpmulti}
\usepackage{algorithm}
\usepackage{algpseudocode}

\usepackage[spanish]{babel}

\addto\captionsenglish{\renewcommand{\definitionname}{Definition}}
\addto\captionsspanish{\renewcommand{\algorithmname}{Algoritmo}}
\addto\captionsspanish{\renewcommand{\definitionname}{Definición}}
\providecommand{\definitionname}{Definition}

\makeatother

\usepackage{babel}
\addto\shorthandsspanish{\spanishdeactivate{~<>}}

\addto\captionsenglish{\renewcommand{\definitionname}{Definition}}
\addto\captionsspanish{\renewcommand{\algorithmname}{Algoritmo}}
\addto\captionsspanish{\renewcommand{\definitionname}{Definición}}
\providecommand{\definitionname}{Definition}

\begin{document}
\title{Taller 2 - Programación Dinámica}
\author{\selectlanguage{spanish}%
David Gutierrez Alarcon,\\ Julian Andres Carrillo Chiquisa,\\ David Alejandro Castillo Chíquiza}
\date{\selectlanguage{spanish}%
\today}
\maketitle

\begin{abstract}
En este documento se presenta el análisis de los dos problemas planteados y sus soluciones mediante el uso de el método de Dividir y Vencer
\end{abstract}

\part*{Análisis y Diseño del Problema}

\section*{Análisis}


\section*{Diseño}


\part*{Algoritmos}


\section*{Evidente recursivo}
	
	\subsection*{Pseudocódigo}
	
	\subsection*{Complejidad}
	
	\subsection*{Invariante}
	
	\subsection*{Notas de Implementación}

\section*{Memorizado}

	\subsection*{Pseudocódigo}
	
	\subsection*{Complejidad}
	
	\subsection*{Invariante}
	
	\subsection*{Notas de Implementación}

\section*{Bottom-up}

	\subsection*{Pseudocódigo}
	
	\subsection*{Complejidad}
	
	\subsection*{Invariante}
	
	\subsection*{Notas de Implementación}

\part*{Comparación}


\end{document}